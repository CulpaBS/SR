\documentclass[11pt]{article}
\usepackage[a4paper, hmargin={2.8cm, 2.8cm}, vmargin={2.5cm, 2.5cm}]{geometry}
\usepackage{eso-pic} % \AddToShipoutPicture
\usepackage{graphicx} % \includegraphics
\usepackage{framed}
\usepackage[utf8]{inputenc}
\usepackage{hyperref}

%% Change `ku-farve` to `nat-farve` to use SCIENCE's old colors or
%% `natbio-farve` to use SCIENCE's new colors and logo.
\def \ColourPDF {include/nat-farve}

%% Change `ku-en` to `nat-en` to use the `Faculty of Science` header
\def \TitlePDF   {include/nat-en}  % University of Copenhagen

\title{
  \vspace{3cm}
  \Huge{Implementing Map-Scan Fusion in the Futhark Compiler} \\
  \Large{Bachelor project}
}

\author{
  \Large{Brian Spiegelhauer}
  \\ \texttt{brianspieg@gmail.com} \\ \\
   \Large{William Jack Lysgaard Sprent}
  \\ \texttt{bsprent@gmail.com} \\
}

\date{
    \today
}

\begin{document}


\AddToShipoutPicture*{\put(0,0){\includegraphics*[viewport=0 0 700 600]{\ColourPDF}}}
\AddToShipoutPicture*{\put(0,602){\includegraphics*[viewport=0 600 700 1600]{\ColourPDF}}}

\AddToShipoutPicture*{\put(0,0){\includegraphics*{\TitlePDF}}}

\clearpage\maketitle
\thispagestyle{empty}

\newpage

\tableofcontents

\newpage

\section{Abstract}
\section{Introduction}
\textit{NOTE: the contents of this section is lifted from our synopsis and is probably placeholder}\\
The Futhark language is a functional programming with which the main idea is to allow for the expression of sufficiently complex programs while keeping complexity to a level where programs can be aggressively optimised and have their parallelism exploited \cite{futharkdoc}.

The Futhark compiler already supports a range of fusion optimisations \cite{T2Fusion}, but does not currently support fusion between \texttt{Map} and \texttt{Scan} statements.

For our project we will explore the possibility of implementing Map-Scan fusion into the Futhark compiler, and will examine the performance benefits (if any) of performing such optimisations.

\subsection{Motivation}
Fusion has the ``[..] potential to optimize both the memory hierarchy time overhead and, sometimes asymptotically, the space requirement" \cite{T2Fusion}. Hence the main motivation for adding Map-Scan fusion capabilities to the optimiser of the Futhark compiler, is the potential for enabling performance increases for some Futhark programs.

\subsection{Tasks}
The project can be divided into three main tasks:
\begin{enumerate}
    \item Gain an understanding of logical reasoning behind fusion optimisations on Second Order Array Combinators.
    \item Read and understand the relevant parts of the Futhark compiler required to make the necessary changes in the compiler.
    \item Modify all modules of the Futhark compiler necessary to implement the Map-Scan fusion itself.
\end{enumerate}
At first sight, these tasks look fairly straight forward. However, we expect that the main difficulties of this project lie within unforeseen roadblocks we will run into when modifying the codebase.

\section{Background Information}
Describe some relevant background info for how SOACs are parallelly computed - relevant to why Scanomap is smart. IMPORTANT: WHY MEMORY MANAGEMENT IS VERY IMPORTANT ON GPU\\

\subsection{Futhark}
As descriped in Troels Henriksens master thesis \cite{MasterTroels} the language $\mathcal{L}_0$ later renamed Futhark is in a sense "sufficient", in that it is Turing-complete, and can express imperative style loops with do-loops. However Futhark is ment to use second-order array combinations (SOACs) to do bulk operations on arrays instead of using the do-loops. In this sections the reasoning behind using SOACs will be explained by showing the difference in their computation when done sequentially vs. parallelly.
%Chunking
\subsection{SOACs}
Both \texttt{map} and \texttt{scan} are defined as SOACs -- or Second Order Array Combinators. Hence they have no free variables, take first-order functions as arguments, and output first-order
 functions whose domains are arrays of the domain of the input. Furthermore, in Futhark, these array inputs and outputs are tuples of arrays, and not arrays of tuples. Working with SOACs allows for some free assumptions to be made which turns out to be useful in regards to both parallisation and optimisation.

% Definition of SOACs
% Why are they useful wrt. parallelisation
% Soacs with tuples in Futhark
\subsubsection{Map}
The $\texttt{map} \: f \: a$ function, has the very simple definition of taking a function $f \: : \: \alpha \to \beta$ and returning a function $\mathtt{map} \:f \: : \: [\alpha] \to [\beta]$  which
 applies $f$ to every element of an input array, $a$.  This gives us the type signature of \texttt{map},
$$\mathtt{map} \: f \: a \: :  \: (\alpha \to \beta) \to [\alpha] \to [\beta]\mathnormal{.}$$
And the semantic definition of \texttt{map},
$$\mathtt{map} \: f \: a \: =  \: [f(a_0), f(a_1), ..., f(a_{n-1})]\mathnormal{.}$$
Having no free variables, means that each result $f(a_i)$ \textit{only} depends on the corresponding element $a_i$. This makes \texttt{map}s fantastic for parallelisation as once the
 degree of parallism reaches the size of $a$, $\mathtt{map} \: f \: a$ can be potentially be computed in a single parrallel step, or $c$ steps for a chunk size of $c$.

% INSERT COOL FIGURE OF MAP BEING COMPUTED 
% How do we compute it parallely
% Timecomplexity?

\subsubsection{Scan}
% what is a scan
$\texttt{scan} \: \odot \: e \: a$ takes the binary, associative function $\odot \: : \: \alpha \to \alpha \to \alpha$ and returns a function
 $\mathtt{scan} \:\odot \: : \: \alpha \to [\alpha] \to [\alpha]$ which
 computes the $\odot$ prefixes of an input array $a$ starting with a neutral element, $e$. Overall, \texttt{scan} has the type signature,
$$\mathtt{scan} \: \odot \: e \: a \: : \:(\alpha \to \alpha \to \alpha) \to \alpha \to [\alpha] \to [\alpha]\mathnormal{.}$$
And when computing \texttt{scan} with the function $\odot$, the array $a$, and neutral element $e$ we get,
$$\mathtt{scan} \: \odot \: e \: a \: = [e \odot a_0, e \odot a_0 \odot a_1, ..., e \odot a_0 \odot ... \odot a_{n-1}]\mathnormal{.}$$
% How do we compute it parallely
Computing such a \texttt{scan} is not as simple as with a \texttt{map} as each prefix $a_0 \odot ... \odot a_i$ obviously depends on the previous prefix $a_0 \odot ... \odot a_{i-1}$. Hence, 
the associativity of $\odot$ is vital as it means that this dependency does not force computation order, and partial results can be computed independently and combined.
% How scan is computed in Futhark with fancy diagram 

% Timecomplexity?

\subsection{Parallel Computations on GPGPUs}
%% Spørg troels om baggrundsinfo

\subsubsection{Memory Conditions}
Small cache high miss penalty.


\section{Map-Scan Fusion}
Describe Map-Scan fusion on multiple levels.
% Naive approach
% Why?
\subsection{Scanomap}
% Why do we need the scanomap construction?
What is the Scanomap construction. What are its semantics, and why is it used.
Show equivalence between a Scanomap and a composition of a map and a scan - similar to showing redomap results from a reduce . map.
\subsection{Necessary Conditions}
% Copy from T2/Troels.
What are the conditions for a scan map fusion. When can we fuse, and when can we not fuse.
Our Scanomap supports carrying outputs from the map - explain why and how.
\subsection{Fusing Scanomap}
% What can we not reduce with
%horizontally
%maposcanomap
\subsection{Fusion Strategy/Example}
Walk through the entire fusion process concisely - or similarly summarize the relevant T2 Graph reduction fusion paper rooted in scan.
Show the fusion of an example instance of maps and scans through dependency graphs etc.

\section{Implementation}

What is the state of the Futhark codebase at project start. How is Scan currently handled. What is already there, and what do we need to implement.\\
Our solution is closely related to how Redomap fusion is handled. How does it differ.\\
What parts of the code have we touched and why.\\
Describe how different parts of the Map-Scan fusion is done - e.g. describe how function composition is implemented and so on.


\section{Benchmarking and Testing}
How have we tested our implementation. Does it work? Why?\\
How does the performance of a fused program compare with a non-fused program both sequentially and parallelly. Why?

\section{Conclusion}
\newpage

\bibliographystyle{unsrt}
\bibliography{lit}

\end{document}