\documentclass[11pt]{article}
\usepackage[a4paper, hmargin={2.8cm, 2.8cm}, vmargin={2.5cm, 2.5cm}]{geometry}
\usepackage{eso-pic} % \AddToShipoutPicture
\usepackage{graphicx} % \includegraphics
\usepackage{framed}
\usepackage[utf8]{inputenc}
\usepackage{hyperref}

%% Change `ku-farve` to `nat-farve` to use SCIENCE's old colors or
%% `natbio-farve` to use SCIENCE's new colors and logo.
\def \ColourPDF {include/nat-farve}

%% Change `ku-en` to `nat-en` to use the `Faculty of Science` header
\def \TitlePDF   {include/nat-en}  % University of Copenhagen

\title{
  \vspace{3cm}
  \Huge{Implementing Map-Scan Fusion in the Futhark Compiler} \\
  \Large{Bachelor project}
}

\author{
  \Large{Brian Spiegelhauer}
  \\ \texttt{brianspieg@gmail.com} \\ \\
   \Large{William Jack Lysgaard Sprent}
  \\ \texttt{bsprent@gmail.com} \\
}

\date{
    \today
}

\begin{document}


\AddToShipoutPicture*{\put(0,0){\includegraphics*[viewport=0 0 700 600]{\ColourPDF}}}
\AddToShipoutPicture*{\put(0,602){\includegraphics*[viewport=0 600 700 1600]{\ColourPDF}}}

\AddToShipoutPicture*{\put(0,0){\includegraphics*{\TitlePDF}}}

\clearpage\maketitle
\thispagestyle{empty}

\newpage

\tableofcontents

\newpage

\section{Abstract}
\section{Introduction}
\textit{NOTE: the contents of this section is lifted from our synopsis and is probably placeholder}\\
The Futhark language is a functional programming with which the main idea is to allow for the expression of sufficiently complex programs while keeping complexity to a level where programs can be aggressively optimised and have their parallelism exploited \cite{futharkdoc}.

The Futhark compiler already supports a range of fusion optimisations \cite{T2Fusion}, but does not currently support fusion between \texttt{Map} and \texttt{Scan} statements.

For our project we will explore the possibility of implementing Map-Scan fusion into the Futhark compiler, and will examine the performance benefits (if any) of performing such optimisations.

\subsection{Motivation}
Fusion has the ``[..] potential to optimize both the memory hierarchy time overhead and, sometimes asymptotically, the space requirement" \cite{T2Fusion}. Hence the main motivation for adding Map-Scan fusion capabilities to the optimiser of the Futhark compiler, is the potential for enabling performance increases for some Futhark programs.

\subsection{Tasks}
The project can be divided into three main tasks:
\begin{enumerate}
    \item Gain an understanding of logical reasoning behind fusion optimisations on Second Order Array Combinators.
    \item Read and understand the relevant parts of the Futhark compiler required to make the necessary changes in the compiler.
    \item Modify all modules of the Futhark compiler necessary to implement the Map-Scan fusion itself.
\end{enumerate}
At first sight, these tasks look fairly straight forward. However, we expect that the main difficulties of this project lie within unforeseen roadblocks we will run into when modifying the codebase.

\section{Parallel Computation / Background info}
Describe some relevant background info for how SOACS are parallelly computed - relevant to why Scanomap is smart. IMPORTANT: WHY MEMORY MANAGEMENT IS VERY IMPORTANT ON GPU\\

\subsection{SOACS}
Describe how SOACS - perhaps maps and scans - are computed sequentially vs. parallelly. \\


As descriped in Troels Henriksens master thesis \cite{MasterTroels} the language $\mathcal{L}_0$ later renamed Futhark is in a sense "sufficient", in that it is Turing-complete, and can express imperative style loops with do-loops. However Futhark is ment to use second-order array combinations (SOACS) to do bulk operations on arrays instead of using the do-loops. In this sections the reasoning behind using SOACS will be explained by showing the difference in their computation when done sequentially vs. parallelly.

\subsection{In Futhark}
Describe how Futhark works with tuples of arrays.
How Futhark splits lists into chunks and combines sequential and parallel methods to compute SOACS.
\section{Map-Scan Fusion}
Describe Map-Scan fusion on multiple levels.
\subsection{Scanomap}
What is the Scanomap construction. What are its semantics, and why is it used.
Show equivalence between a Scanomap and a composition of a map and a scan - similar to showing redomap results from a reduce . map.
\subsection{Necessary Conditions}
What are the conditions for a scan map fusion. When can we fuse, and when can we not fuse.
Our Scanomap supports carrying outputs from the map - explain why and how.
\subsection{Fusing Scanomap}
\subsection{The Fusion Process}
Walk through the entire fusion process concisely - or similarly summarize the relevant T2 Graph reduction fusion paper rooted in scan.
Show the fusion of an example instance of maps and scans through dependency graphs etc.

\section{Implementation}
What is the state of the Futhark codebase at project start. How is Scan currently handled. What is already there, and what do we need to implement.\\
Our solution is closely related to how Redomap fusion is handled. How does it differ.\\
What parts of the code have we touched and why.\\
Describe how different parts of the Map-Scan fusion is done - e.g. describe how function composition is implemented and so on.

\section{Benchmarking and Tests}
How have we tested our implementation. Does it work? Why?\\
How does the performance of a fused program compare with a non-fused program both sequentially and parallelly. Why?

\section{Conclusion}
\newpage

\bibliographystyle{unsrt}
\bibliography{lit}

\end{document}